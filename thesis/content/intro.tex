\section{Introduction}
\label{sec:introduction}

When evaluating a programming language for use on a project, we as programmers are faced with an ever-growing array of choices. These choices range from long-lived, well-established languages such as C or Fortran, to the very latest offerings such as Julia or Swift. Selecting a language is a process that is generally rooted in a combination of factors: suitability to the target platform, performance, expressiveness, and (often most importantly) the programmer's familiarity and comfort with the language.

This thesis will be a comparison of the relative strengths of the Rust programming language when compared to C and C++. To explore this, the problem of large-scale string matching will be examined with a focus on matching within DNA-like strings.

\subsection{String Matching}

The topic of string matching has long been a popular area of research in computer science. Before the paper by Knuth, Morris and Pratt in 1977~\cite{knuth.morris.pratt.1977} there was already considerable work being done. In the same year, Boyer and Moore published an improvement over the Knuth/Morris/Pratt algorithm with enhancements such as starting the match from the tail of the pattern rather than the head, and allowing for greater right-ward jumps through the string being searched~\cite{boyer.moore.1977}.

But string matching means extensive reading and manipulation of blocks of allocated memory, which can lead to program errors and vulnerabilities. A large percentage of security vulnerabilities discovered in programs are traced back to memory-related issues; in~\cite{cimpanu.2020}, Google software engineers are quoted as attributing roughly 70\% of serious security bugs in Chrome to memory management and safety bugs. The article goes on to report that analysis from Microsoft echoes this number. As such, a process to write programs that are more stable and secure must include careful consideration of memory-related challenges.

\subsection{DNA Strings}

For this study, string matching will be applied to the problem of finding sub-sequences within DNA strings. DNA sequence strings have some interesting properties, in that they can be \textit{extremely} long but at the same time the alphabet is limited to just four characters (``A'', ``C'', ``G'' and ``T'').

\subsection{Comparison Bases}
\label{subsec:comparison}

Solutions to the problem will be developed in three languages: C, C++, and Rust. Each language will be evaluated against the others on three bases:

\begin{enumerate}
\item \textbf{Execution speed}: Overall run-times for each solution will be gathered using existing timer mechanisms. Time-measurements will be somewhat coarse, as overhead operations such as I/O will be included in the times.
\item \textbf{Readability and expressiveness}: Each solution will also be measured on several source-level metrics in an effort to evaluate the expressiveness and clarity of the code.
\item \textbf{Energy efficiency}: Energy usage will be measured for each solution using the Running Average Power Limit (RAPL) tools available to Intel processors.
\end{enumerate}

These three bases cover modern concerns in software development: the general performance of an application, the readability/maintainability of the application, and the power consumption of the system running the application. Where the first two criteria are well-known and common, the last basis is used based on steadily-growing concern over power consumption in the data center industry and in the mobile computing field~\cite{pereira.et.al.2017}.
