\section{Details of Experiments}
\label{sec:experiments}

To explore these hypotheses a wide range of experiments were performed in the three chosen languages (C, C++, Rust), with the programs being run under a ``harness'' application that measures various power and energy metrics. Additional tools were used to examine the running programs for memory leaks, as well as measure aspects of the source code itself.

\subsection{Definitions and Measurements}

To begin with, some concepts and terms will be defined.

\subsubsection{RAPL (Running Average Power Limit)}

\subsubsection{SLOC (Source Lines of Code)}

\subsubsection{Measurements}

\subsection{Languages Under Consideration}

As mentioned in section \ref{subsec:comparison}, the experiments were performed on three languages. The languages were chosen for their commonalities as well as their differences:

\begin{itemize}
\item Each language is compiled to machine code (none are interpreted or make use of virtual machines)
\item Each is widely regarded with their respective communities as being viable for systems-level programming
\item Each takes a somewhat different view on memory management
\end{itemize}

\subsubsection{C}

The C programming language is the oldest and most-established of the three languages. Originally designed in the early 1970's by Dennis Ritchie, it remains a very widely-used and influential language since its first appearance in 1972. Since 1989, it has been standardized by both ANSI (the American National Standards Institute) and by the International Organization for Standardization (ISO).

\subsubsection{C++}

C++ was developed initially as an extension of C, by Bjarne Stroustrup while working at AT\&T Bell Labs. It first appeared in 1985 and was initially standardized in 1998. Initially envisioned as ``C with Classes'', the language has been significantly expanded over the years to include many more features while still maintaining low-level memory accessibility. C++ attempts to offer more expressive, concise coding than C, with many of C's memory-management concerns dealt with automatically by class constructors and destructors.

\subsubsection{Rust}

Rust is the newest of the three languages, having first appeared in 2010 (as covered in detail in section \ref{sec:rust}). Rust brings a promise of expressiveness equal to or greater than C++, with greater safety in the areas of memory management and ownership.

\subsection{Selected Algorithms}

Here, the various algorithms that were chosen will be covered in detail.

\subsubsection{Knuth, Morris, and Pratt}

\subsubsection{Boyer and Moore}

\subsubsection{Shift-Or}

\subsubsection{Aho and Corasick}

\subsubsection{Other Algorithms Considered}

\subsection{Degrees of Optimization}

\subsection{Data Used}

\subsubsection{Random vs. Existing Data}

\subsubsection{Method of Generation}

\subsubsection{Nature of the Data Used}

\subsection{Testing Platform}

\subsubsection{Hardware Specifications}

\subsubsection{Operating System and Configuration}

\subsubsection{Language Compilers and Tools}
